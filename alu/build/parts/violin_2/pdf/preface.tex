\documentclass[11pt]{article}
\usepackage{fontspec}
\usepackage[utf8]{inputenc}
\usepackage{xunicode}
\setmainfont{Bell MT}
\usepackage[paperwidth=8.5in,paperheight=11in,margin=1in,headheight=0.0in,footskip=0.5in,includehead,includefoot,portrait]{geometry}
\usepackage[absolute]{textpos}
\TPGrid[0.5in, 0.25in]{23}{24}
\parindent=0pt
\parskip=12pt
\usepackage{nopageno}
\usepackage{graphicx}
\graphicspath{ {./images/} }
\usepackage{amsmath}
\usepackage{tikz}
\newcommand*\circled[1]{\tikz[baseline=(char.base)]{
            \node[shape=circle,draw,inner sep=1pt] (char) {#1};}}

\begin{document}

\begin{center}
\huge PERFORMANCE NOTES
\end{center}
\begingroup
\begin{center}

\leftskip0.25in
\pmb{String Contact Points} : The indications of string contact positions such as $sul \ tasto$ (abbreviated as $T$), $sul \ ponticello$ (abbreviated as $P$), $extreme \ sul \ tasto$ (abbreviated as $XT$), etc. should be considered as points along the continuum of the length string. The performer should make an effort to smoothly transition from one position to the next throughout the duration of the passage covered by the arrow-demarcated dashed line. When this arrow is not present, the performer should default to an $ordinario$ position.
\rightskip\leftskip
\phantom{text} \hfill \phantom{()}

\leftskip0.25in
\pmb{Bow Contact Points} : In various passages throughout this piece, there is notation which represents the point at which the bow is touched as it is drawn across the string. These positions are written as fractions where \( \frac{0}{7} \) and  \( \frac{0}{5} \) represent $au \ talon$ and \( \frac{7}{7} \) and \( \frac{5}{5} \) represent $punta \ d'arco$. For the duration of the note to which these fractions are attached, the performer should draw the bow at a constant speed, moving toward the destination point indicated on the following note. Bowings are provided. Passages without these indications should be bowed at the performer's discretion.
\rightskip\leftskip
\phantom{text} \hfill \phantom{()}

\leftskip0.25in
\pmb{Bow Rotation Indications} : \circled{1} $col \ legno \ tratto$ is abbreviated as $clt.$ and \circled{2} $col \ legno \ batutto$ is abbreviated as $clb.$. When these abbreviations are not present, the performer should default to ordinary $crine$ bowing techniques.
\rightskip\leftskip
\phantom{text} \hfill \phantom{()}

\leftskip0.25in
\pmb{Accidentals} : After temporary accidentals, cancellation marks are printed also in the following measure (for notes in the same octave) and, in the same measure, for notes in other octaves, but they are printed again if the same note appears later in the same measure, except if the note is immediately repeated.
\rightskip\leftskip
\phantom{text} \hfill \phantom{()}

\leftskip0.25in
\pmb{Miscellaneous} : \circled{1} Tremoli should be performed as fast as possible and not as a measured subdivision of the duration to which they are attached. \circled{2} Diamond note heads represent a left hand finger pressure of a natural harmonic. \circled{3} Half-harmonic finger pressure is shown with a diamondhalf-filled with black for short durations and a diamond open on one end for long durations. \circled{4}  Vibrato is indicated with a wavy line above the staff. \circled{5} A wavy line next to a note head indicates extreme, tight, glissando.
\rightskip\leftskip
\phantom{text} \hfill \phantom{()}

\end{center}
\endgroup

\vspace*{9\baselineskip}

\begin{center}
\textit{Alu} was composed in partial fulfillment of the requirements for the degree of Doctor of Philosophy in the subject of Music Composition at the University of Iowa.
\end{center}

\vspace*{23\baselineskip}

\begin{center}
duration: c. 20'
\end{center}

\end{document}
