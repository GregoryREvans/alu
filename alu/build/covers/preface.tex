\documentclass[11pt]{article}
\usepackage{fontspec}
\usepackage[utf8]{inputenc}
\usepackage{xunicode}
\setmainfont{Bell MT}
\usepackage[paperwidth=11in,paperheight=17in,margin=1in,headheight=0.0in,footskip=0.5in,includehead,includefoot,portrait]{geometry}
\usepackage[absolute]{textpos}
\TPGrid[0.5in, 0.25in]{23}{24}
\parindent=0pt
\parskip=12pt
\usepackage{nopageno}
\usepackage{graphicx}
\graphicspath{ {./images/} }
\usepackage{amsmath}
\usepackage{tikz}
\newcommand*\circled[1]{\tikz[baseline=(char.base)]{
            \node[shape=circle,draw,inner sep=1pt] (char) {#1};}}

\begin{document}

\begin{textblock}{23}(0, 1)
\begin{center}
\huge FOREWORD
\end{center}
\end{textblock}

\vspace*{0.25\baselineskip}

\begingroup

\begin{center}
The Eggja stone was found with the written side downwards over a man's grave which is dated to the period 650-700 C.E. Having as many as 200 runes, it is the longest known inscription in the Elder Futhark, but certain runes are transitional towards the Younger Futhark. It is generally agreed that it is written in stylized poetry and in a partly metrical form containing a protection for the grave and the description of a funerary rite. However, there are widely diverging interpretations about certain details. The following transcription mostly copies the graphic analysis provided by Ottar Grønvik (1985). Some of the individual characters are unclear, and other analyses may disagree with certain parts:
\end{center}

\begin{textblock}{7.333}(0, 4.5)

\leftskip0.5in
\setmainfont{Noto Sans Runic}
ᚾᛁᛋᛋᛟᛚᚢᛋᛟᛏᚢᚴᚾᛁᛋᚼᚴᛋᛖᛋᛏᚼᛁᚾᛋᚴᛟᚱᛁᚾ \\
\hfill \break
ᚾᛁᛚᚼᛁᚷᛁᛗᚨᛦᚾᚼᚴᛞᚨᚾᛁᛋᚾᛁᚦᚱᛁᚾᛦ\\
\hfill \break
ᚾᛁᚹᛁᛚᛏᛁᛦᛗᚨᚾᛦᛚᚼᚷᛁ\\
  \hfill \break
ᚨᛚᚢᛗᛁᛋᚢᚱᚴᛁ\\
\rightskip\leftskip
\phantom{text} \hfill

\end{textblock}

\begin{textblock}{7.333}(7.8333, 4.5)
\setmainfont{Bell MT}
Ni's sólu sótt ok ni saxe stæin skorinn. \\
\hfill \break
Ni (læggi) mannr nækðan, is ni\Thorn \ rinnr, \\
\hfill \break
Ni viltir mænnr læggi ax. \\
\hfill \break
Alu misyrki. \\
\end{textblock}

\begin{textblock}{7.333}(15.666, 4.5)
It is not touched by the sun and the stone is not scored by an (iron) knife. \\
No man may lay (it) bare, when the waning moon runs (across the heavens). \\
Misguided men may not lay (the stone) aside. \\
\hfill \break
Protection against the wrong-doer.
\end{textblock}

\endgroup

\vspace*{8\baselineskip}

\begin{center}
\huge INSTRUMENTATION
\end{center}

\hspace*{1cm} Flute (doubling Bass Flute)
\\
\hspace*{1cm} Oboe
\\
\hspace*{1cm} Clarinet in B-flat (doubling Bass Clarinet)
\\
\hspace*{1cm} Bassoon
\\
\hspace*{1cm} Horn in F
\\
\hspace*{1cm} Trumpet in C
\\
\hspace*{1cm} Tenor Trombone
\\
\hspace*{1cm} Tuba
\\
\hspace*{1cm} Percussion 1
\\
\hspace*{2cm} Instruments:
\\
\hspace*{3cm} Bass Drum [x1]
\\
\hspace*{3cm} Tom-toms [x2]
\\
\hspace*{3cm} Congas [x2]
\\
\hspace*{3cm} Bongos [x2]
\\
\hspace*{3cm} Extra Music Stand (empty for performance)
\\
\hspace*{2cm} Implements:
\\
\hspace*{3cm} Mallets suitable for all drums
\\
\hspace*{3cm} Bow
\\
\hspace*{1cm} Percussion 2
\\
\hspace*{2cm} Instruments:
\\
\hspace*{3cm} Bass Drum [x1]
\\
\hspace*{3cm} Tom-toms [x2]
\\
\hspace*{3cm} Congas [x2]
\\
\hspace*{3cm} Bongos [x2]
\\
\hspace*{3cm} Extra Music Stand (empty for performance)
\\
\hspace*{2cm} Implements:
\\
\hspace*{3cm} Mallets suitable for all drums
\\
\hspace*{3cm} Bow
\\
\hspace*{1cm} Piano
\\
\hspace*{1cm} Violin x2
\\
\hspace*{1cm} Viola
\\
\hspace*{1cm} Violoncello
\\
\hspace*{1cm} Contrabass

\vspace*{1.25\baselineskip}

\begin{center}
\huge PERFORMANCE NOTES
\end{center}
\begingroup
\begin{center}

\leftskip0.25in
\pmb{String Contact Points} : The indications of string contact positions such as $sul \ tasto$ (abbreviated as $T$), $sul \ ponticello$ (abbreviated as $P$), $extreme \ sul \ tasto$ (abbreviated as $XT$), etc. should be considered as points along the continuum of the length string. The performer should make an effort to smoothly transition from one position to the next throughout the duration of the passage covered by the arrow-demarcated dashed line. When this arrow is not present, the performer should default to an $ordinario$ position.
\rightskip\leftskip
\phantom{text} \hfill \phantom{()}

\leftskip0.25in
\pmb{Bow Contact Points} : In various passages throughout this piece, there is notation which represents the point at which the bow is touched as it is drawn across the string. These positions are written as fractions where \( \frac{0}{7} \) and  \( \frac{0}{5} \) represent $au \ talon$ and \( \frac{7}{7} \) and \( \frac{5}{5} \) represent $punta \ d'arco$. For the duration of the note to which these fractions are attached, the performer should draw the bow at a constant speed, moving toward the destination point indicated on the following note. Bowings are provided. Passages without these indications should be bowed at the performer's discretion.
\rightskip\leftskip
\phantom{text} \hfill \phantom{()}

\leftskip0.25in
\pmb{Bow Rotation Indications} : \circled{1} $col \ legno \ tratto$ is abbreviated as $clt.$ and \circled{2} $col \ legno \ batutto$ is abbreviated as $clb.$. When these abbreviations are not present, the performer should default to ordinary $crine$ bowing techniques.
\rightskip\leftskip
\phantom{text} \hfill \phantom{()}

\pmb{Repeats} : Two unusual repeats are given in the score: one overlapping repeat and one nested repeat. The units of these complex repeats are distinguished by the color of the repeat-bar symbol. The nested repeat should be performed in its entirety each time the outer repeat is played. The overlapped repeat should only be played once, without triggering the repeat which immediately precedes it.
\rightskip\leftskip
\phantom{text} \hfill \phantom{()}

\leftskip0.25in
\pmb{Accidentals} : \circled{1} After temporary accidentals, cancellation marks are printed also in the following measure (for notes in the same octave) and, in the same measure, for notes in other octaves, but they are printed again if the same note appears later in the same measure, except if the note is immediately repeated. \circled{2} At times throughout the score, justly tuned intervals are indicated by the use of Helmholtz-Ellis notation combined with cent deviations from equal temperament for use with an electronic tuner. When no example pitch is given with the cent deviation, the mark is a deviation of the nearest ``standard'' accidental. If the performers wish to interpret the score without cent-tuning, the approximation of pitches to the nearest semi-tone is acceptable. When Helmholtz-Ellis notation is not given, the pitches are to be played as usual. The accidentals for Justly-intoned pitches are always present before the note head. \circled{3} Some pitch content is derived from non-octaving equal tempered scales (for instance, 9 even divisions of the ratio 7/2). When these pitches are to be played, deviation from 12-Tone-Equal-Temperament is given in cents above with the frequency in hertz shown below.
\rightskip\leftskip
\phantom{text} \hfill \phantom{()}

\leftskip0.25in
\pmb{Miscellaneous} : \circled{1} Tremoli should be performed as fast as possible and not as a measured subdivision of the duration to which they are attached. \circled{2} Diamond note heads represent a left hand finger pressure of a natural harmonic. \circled{3} Half-harmonic finger pressure is shown with a diamondhalf-filled with black for short durations and a diamond open on one end for long durations. \circled{4}  Vibrato is indicated with a wavy line above the staff. \circled{5} A wavy line next to a note head indicates extreme, tight, glissando.
\rightskip\leftskip
\phantom{text} \hfill \phantom{()}

\end{center}
\endgroup

\vspace*{9\baselineskip}

\begin{center}
\textit{ASPLEDON alu} was composed for the JACK Quartet as part of the 2022 residency at the University of Iowa.
\end{center}

\vspace*{23\baselineskip}

\begin{center}
duration: c. 11'
\end{center}

\end{document}
